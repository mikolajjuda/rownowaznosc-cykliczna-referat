\documentclass{scrartcl}
\usepackage[a4paper, margin=1in]{geometry}
\usepackage{relsize}
\usepackage[T1]{fontenc}
\usepackage[polish]{babel}
\usepackage{listings}
\usepackage{csquotes}
\usepackage{mathtools, amsmath, amsthm, amsfonts, amssymb}
\newcommand{\Mod}[1]{\ (\mathrm{mod}\ #1)}
\usepackage{braket}

\usepackage{hyperref}
\usepackage[capitalize,nameinlink]{cleveref}
\usepackage{url}

\usepackage[dvipsnames]{xcolor}
\hypersetup{
colorlinks=true,
linkcolor=BrickRed,
citecolor=Green,
urlcolor=blue,
frenchlinks=true,
pdftitle={Równoważność cykliczna ciągów},
pdfpagemode=FullScreen,
}

\usepackage[backend=biber, sorting=ynt]{biblatex}
\addbibresource{bibliografia.bib}

\addtokomafont{title}{\normalfont\bfseries}
\addtokomafont{disposition}{\rmfamily}

\title{Równoważność cykliczna ciągów}
\subtitle{Definicja problemu i przedstawienie rozwiązań}
\date{2023}
\author{Mikołaj Juda}

\renewcommand\qedsymbol{\(\blacksquare\)}

\theoremstyle{definition}
\newtheorem{cyclic_equivalence_1}{Definicja}[section]
\theoremstyle{definition}
\newtheorem{cyclic_equivalence_2}
[cyclic_equivalence_1]{Definicja}
\newtheorem{lemma_k0_values}{Lemat}[section]
\newtheorem{less_k0_values_to_check}[lemma_k0_values]{Wniosek}

\setlength\parindent{0pt}

\begin{document}
\relscale{1.4}
\maketitle
\begin{abstract}
    W referacie przedstawiono
    problem równoważności cyklicznej ciągów
    oraz różne algorytmy do jego rozwiązania
    razem z implementacją w języku Python.
    Pokrótce omówiono algorytm naiwny
    oraz algorytm korzystający z wyszukiwania wzorca.
    Przedstawiono również
    szybki algorytmie sprawdzania równoważności
    list cyklicznych Shiloacha(1979)\cite{shiloach1979}
    oraz szczegółowo opisano dowód jego poprawności
    i analizę złożoności obliczeniowej.
\end{abstract}
\tableofcontents
\pagebreak
\section{Definicja problemu}
Dane są dwa ciągi
\(A=(a_0,\ldots,a_{n-1})\) oraz \(B=(b_0,\ldots,b_{n-1})\)
długości \(n\).\linebreak
\(A\) i \(B\) są \emph{równoważne cyklicznie}
(\(A\equiv B\)),
gdy są równe w sensie list cyklicznych tzn.
\begin{cyclic_equivalence_1}
    \label{def:cyclic_equivalence_1}
    \[A\equiv B \iff \exists_{k_0\in\mathbb{Z}}
        \forall_{k\in\set{0,\ldots,n-1}}\ a_{(k_0+k)\Mod{n}}=b_k\]
\end{cyclic_equivalence_1}

Dla wygody dalszego zapisu oznaczmy:
\[a_k\coloneq a_{k\Mod{n}},\ b_k\coloneq b_{k\Mod{n}}
    \ \text{dla wszystkich}\ k\ge n\]

Zdefiniujmy \(A_k\) jako listę powstałą z przesunięcia cyklicznego
ciągu \(A\) takiego, że \(a_k\) jest pierwszym elementem ciągu \(A_k\).
Analogicznie dla \(B_k\).\footnote{\(A_0=[a_0,\ldots,a_{n-1}]\),
oraz \(B_0=[b_0,\ldots,b_{n-1}]\)}
\begin{align*}
    A_k=[a_{k},\ldots,a_n,a_0,\ldots,a_{k-1}] \\
    B_k=[b_{k},\ldots,b_n,b_0,\ldots,b_{k-1}]
\end{align*}

Definicję \cref{def:cyclic_equivalence_1} można
przedstawić równoważnie jako:
\begin{cyclic_equivalence_2}
    \[A\equiv B \iff \exists_{k_0\in\mathbb{Z}}
        \ A_{k_0}=B_0\]
\end{cyclic_equivalence_2}

Podsumowując, problem brzmi:
\enquote{Czy istnieje takie przesunięcie cykliczne jednego ciągu,
    że jest po nim równy drugiemu ciagowi?}
\pagebreak
\section{Algorytm naiwny}
\subsection{Opis}
Z \hyperref[def:cyclic_equivalence_1]
{Definicji \ref*{def:cyclic_equivalence_1}}
można łatwo zauważyć, że
\begin{lemma_k0_values}
    \label{lem:lemma_k0_values}
    Jeżeli nie istnieje \(k_0\in\set{0,\ldots, n-1}\) spełniające warunek:
    \[\forall_{k\in\set{0,\ldots,n-1}}\ a_{k_0+k}=b_k\]
    to nie istnieje \(k_0\in\mathbb{Z}\) spełniające ten warunek.
\end{lemma_k0_values}
\begin{proof}
    Oczywiste.
\end{proof}
\begin{less_k0_values_to_check}
    \label{cor:less_k0_values_to_check}
    Żeby ustalić istnienie \(k_0\) z
    \hyperref[def:cyclic_equivalence_1]
    {Definicji \ref*{def:cyclic_equivalence_1}}
    wystarczy sprawdzić czy
    \[\exists_{k_0\in\set{0,\ldots,n-1}}
        \forall_{k\in\set{0,\ldots,n-1}}\ a_{k_0+k}=b_k\]
\end{less_k0_values_to_check}

Algorytm naiwny sprawdza dla każdego \(c\in\set{0,\ldots,n-1}\)
czy \[\forall_{k\in\set{0,\ldots,n-1}}\ a_{c+k}=b_k\]
Jeśli trafi na \(c\) spełniające warunek to zwraca \lstinline|True|,
w przeciwnym wypadku zwraca \lstinline|False|.
Algorytm ma złożoność kwadratową.\cite{wazniakmimuw}
\subsection{Implementacja}
\lstinputlisting[language=Python]{naive.py} %TODO try minted
\pagebreak
\printbibliography[heading=bibintoc]
\end{document}
