\documentclass{scrartcl}
\usepackage[a4paper, margin=1in]{geometry}
\usepackage{relsize}
\usepackage[T1]{fontenc}
\usepackage[polish]{babel}
\usepackage{minted}
\usemintedstyle{autumn}
\usepackage{csquotes}
\usepackage{mathtools, amsmath, amsthm, amsfonts, amssymb}
\newcommand{\Mod}[1]{\ (\mathrm{mod}\ #1)}
\usepackage{braket}

\usepackage{hyperref}
\usepackage[capitalize,nameinlink]{cleveref}
\usepackage{url}

\usepackage[dvipsnames]{xcolor}
\hypersetup{
colorlinks=true,
linkcolor=BrickRed,
citecolor=Green,
urlcolor=blue,
frenchlinks=true,
pdftitle={Równoważność cykliczna ciągów},
pdfpagemode=FullScreen,
}

\usepackage[backend=biber, sorting=ynt]{biblatex}
\addbibresource{bibliografia.bib}

\addtokomafont{title}{\normalfont\bfseries}
\addtokomafont{disposition}{\rmfamily}

\title{Równoważność cykliczna ciągów}
\subtitle{Definicja problemu i przedstawienie rozwiązań}
\date{2023}
\author{Mikołaj Juda}

\renewcommand\qedsymbol{\(\blacksquare\)}

\theoremstyle{definition}
\newtheorem{cyclic_equivalence_1}{Definicja}[section]
\theoremstyle{definition}
\newtheorem{cyclic_equivalence_2}
[cyclic_equivalence_1]{Definicja}
\newtheorem{lemma_k0_values}{Lemat}[section]
\newtheorem{less_k0_values_to_check}{Wniosek}[section]

\setlength\parindent{0pt}

\begin{document}
\relscale{1.4}
\maketitle
\begin{abstract}
    W referacie przedstawiono
    problem równoważności cyklicznej ciągów
    oraz różne algorytmy do jego rozwiązania
    razem z implementacją w języku Python.
    Pokrótce omówiono algorytm naiwny
    oraz algorytm korzystający z wyszukiwania wzorca.
    Przedstawiono również
    szybki algorytmie sprawdzania równoważności
    list cyklicznych Shiloacha(1979)\cite{shiloach1979}
    oraz szczegółowo opisano dowód jego poprawności
    i analizę złożoności obliczeniowej.
\end{abstract}
\tableofcontents
\pagebreak
\section{Definicja problemu}
Dane są dwa ciągi
\(A=(a_0,\ldots,a_{n-1})\) oraz \(B=(b_0,\ldots,b_{n-1})\)
długości \(n\).\linebreak
\(A\) i \(B\) są \emph{równoważne cyklicznie}
(\(A\equiv B\)),
gdy są równe w sensie list cyklicznych tzn.
\begin{cyclic_equivalence_1}
    \label{def:cyclic_equivalence_1}
    \[A\equiv B \iff \exists_{k_0\in\mathbb{Z}}
        \forall_{k\in\set{0,\ldots,n-1}}\ a_{(k_0+k)\Mod{n}}=b_k\]
\end{cyclic_equivalence_1}

Dla wygody dalszego zapisu oznaczmy:
\[a_k\coloneq a_{k\Mod{n}},\ b_k\coloneq b_{k\Mod{n}}
    \ \text{dla wszystkich}\ k\ge n\]

Zdefiniujmy \(A_k\) jako listę powstałą z przesunięcia cyklicznego
ciągu \(A\) takiego, że \(a_k\) jest
pierwszym elementem ciągu \(A_k\).
Analogicznie dla \(B_k\).\footnote{\(A_0=[a_0,\ldots,a_{n-1}]\),
oraz \(B_0=[b_0,\ldots,b_{n-1}]\)}
\begin{align*}
    A_k=[a_{k},\ldots,a_n,a_0,\ldots,a_{k-1}] \\
    B_k=[b_{k},\ldots,b_n,b_0,\ldots,b_{k-1}]
\end{align*}

Definicję \cref{def:cyclic_equivalence_1} można
przedstawić równoważnie jako:
\begin{cyclic_equivalence_2}
    \[A\equiv B \iff \exists_{k_0\in\mathbb{Z}}
        \ A_{k_0}=B\]
\end{cyclic_equivalence_2}

Podsumowując, problem brzmi:
\enquote{Czy istnieje takie przesunięcie cykliczne jednego ciągu,
    że jest po nim równy drugiemu ciagowi?}
\pagebreak
\section{Algorytm naiwny} \label{sec:naive_alg}
\subsection{Opis}
Z \hyperref[def:cyclic_equivalence_1]
{Definicji \ref*{def:cyclic_equivalence_1}}
można łatwo zauważyć, że
\begin{lemma_k0_values}
    \label{lem:lemma_k0_values}
    Jeżeli nie istnieje
    \(k_0\in\set{0,\ldots, n-1}\) spełniające warunek:
    \[\forall_{k\in\set{0,\ldots,n-1}}\ a_{k_0+k}=b_k\]
    to nie istnieje \(k_0\in\mathbb{Z}\) spełniające ten warunek.
\end{lemma_k0_values}
\begin{proof}
    Oczywiste.
\end{proof}
\begin{less_k0_values_to_check}
    \label{cor:less_k0_values_to_check}
    Żeby ustalić istnienie \(k_0\) z
    \hyperref[def:cyclic_equivalence_1]
    {Definicji \ref*{def:cyclic_equivalence_1}}
    wystarczy sprawdzić czy
    \[\exists_{k_0\in\set{0,\ldots,n-1}}
        \forall_{k\in\set{0,\ldots,n-1}}\ a_{k_0+k}=b_k\]
\end{less_k0_values_to_check}
\vskip 2em

Algorytm naiwny sprawdza dla każdego \(l\in\set{0,\ldots,n-1}\)
czy \[\forall_{k\in\set{0,\ldots,n-1}}\ a_{l+k}=b_k\]
Jeśli trafi na \(l\) spełniające warunek
to mamy \(k_0=l\)
i algorytm zwraca \mintinline{python3}|True|,
w przeciwnym wypadku zwraca \mintinline{python3}|False|.
Algorytm ma złożoność kwadratową.\cite{wazniakmimuw}
\subsection{Implementacja}
\inputminted{python3}{naive.py}
\pagebreak
\section{Algorytm wykorzystujący wyszukiwanie wzorca}
\subsection{Opis}
Utwórzmy listę \[AA=[a_0,\ldots, a_{n-1}, a_0,\ldots, a_{n-1}]\]
i zauważmy, że każdy spójny podciąg \(AA\) o długości \(n\)
rozpoczynający się od indeksu \(k\) ma postać \(A_k\),
czyli jest przesunięciem cyklicznym ciągu \(A\).
Zatem jeśli sprawdzimy czy \(B\) jest spójnym podciągiem \(AA\)
to otrzymamy rozwiązanie problemu równoważnosci cyklicznej.

Można więc wykorzystać tytaj dowolny algorytm wyszukiwania wzorca,
jednakże naiwny algorytm wyszukiwania wzorca sprowadza się do
wcześniej przedstawionego
\hyperref[sec:naive_alg]{algorytmu naiwnego}
i ma złożoność kwadratową.
Wykorzystanie algorytmu wyszukiwania wzorca
o liniowej złożoności obliczeniowej
umożliwia sprawdzenie równoważności cyklicznej
w czasie liniowym.\cite{alg}
Wersja wykorzystujaca algorytm Knutha-Morrisa-Pratta
wykonuje około 5n porównań.\cite{shiloach1979}
\pagebreak
\subsection[Implementacja]{Implementacja\footnote{
        Użyta implementacja algorytmu KMP (z modyfikacjami)
        pochodzi ze strony:\\
        \url{https://www.geeksforgeeks.org/%
            python-program-for-kmp-algorithm%
            -for-pattern-searching-2/}
    }}
\inputminted[fontsize=\normalsize]{python3}{pattern_matching.py}

\pagebreak
\printbibliography[heading=bibintoc]
\end{document}
