\documentclass{scrartcl}
\usepackage[a4paper, margin=1in]{geometry}
\usepackage{relsize}
\usepackage[T1]{fontenc}
\usepackage[polish]{babel}
\usepackage{minted}
\usemintedstyle{autumn}
\usepackage{algpseudocode}
\usepackage{csquotes}
\usepackage{mathtools, amsmath, amsthm, amsfonts, amssymb}
\usepackage{commath}
\newcommand{\Mod}[1]{\ (\mathrm{mod}\ #1)}
\usepackage{braket}
\usepackage{enumitem}

\usepackage{hyperref}
\usepackage[capitalize,nameinlink]{cleveref}
\usepackage{url}

\usepackage[dvipsnames]{xcolor}
\hypersetup{
colorlinks=true,
linkcolor=BrickRed,
citecolor=Green,
urlcolor=blue,
frenchlinks=true,
pdftitle={Równoważność cykliczna ciągów},
pdfpagemode=FullScreen,
}

\usepackage[backend=biber, sorting=ynt]{biblatex}
\addbibresource{bibliografia.bib}

\addtokomafont{title}{\normalfont\bfseries}
\addtokomafont{disposition}{\rmfamily}

\title{Równoważność cykliczna ciągów}
\subtitle{Definicja problemu i przedstawienie rozwiązań}
\date{2023}
\author{Mikołaj Juda}

\renewcommand\qedsymbol{\(\blacksquare\)}

\theoremstyle{definition}
\newtheorem{cyclic_equivalence_1}{Definicja}[section]
\newtheorem{cyclic_equivalence_2}
[cyclic_equivalence_1]{Definicja}

\theoremstyle{remark}
\newtheorem{b_shift_def_remark}{Uwaga}[section]

\theoremstyle{plain}
\newtheorem{lemma_k0_values}{Lemat}[section]
\newtheorem{less_k0_values_to_check}{Wniosek}[section]

\theoremstyle{remark}
\newtheorem{A_i_B_j_all_rotations}{Uwaga}[section]
\newtheorem{equiv_iff_shift_sets_equal}{Uwaga}[section]

\theoremstyle{plain}
\newtheorem{lemma_equiv_if_lex_min_eq}{Lemat}[section]
\newtheorem{lemma_k_shift_greater}
[lemma_equiv_if_lex_min_eq]{Lemat}
\newtheorem{lemma_1_n_in_d_no_equiv}
[lemma_equiv_if_lex_min_eq]{Lemat}
\newtheorem{lemma_niezmiennik}[lemma_equiv_if_lex_min_eq]{Lemat}
\newtheorem{not_storing_sets}{Wniosek}[section]
\newtheorem{termination_theorem}{Twierdzenie}[section]
\newtheorem{correctness_theorem}[termination_theorem]{Twierdzenie}

\theoremstyle{definition}
\newtheorem{def_success_compare}{Definicja}[section]
\newtheorem{def_D_star}[def_success_compare]{Definicja}

\theoremstyle{plain}
\newtheorem{lemma_unsuccesful_comparisons}
[lemma_equiv_if_lex_min_eq]{Lemat}
\newtheorem{lemma_d_star_len}[lemma_equiv_if_lex_min_eq]{Lemat}
\newtheorem{complexity_theorem}[termination_theorem]{Twierdzenie}
\newtheorem{o_complexity}[not_storing_sets]{Wniosek}

\theoremstyle{remark}
\newtheorem{remark_max_comp_input}[equiv_iff_shift_sets_equal]{Uwaga}

\theoremstyle{plain}

\setlength\parindent{0pt}

\begin{document}
\relscale{1.4}
\maketitle
\begin{abstract}
	W referacie przedstawiono
	problem równoważności cyklicznej ciągów
	oraz różne algorytmy do jego rozwiązywania
	razem z implementacją w języku Python.
	Pokrótce omówiono algorytm naiwny
	oraz algorytm korzystający z wyszukiwania wzorca.
	Przedstawiono również
	szybki algorytm sprawdzania równoważności
	list cyklicznych Shiloacha(1979)\cite{shiloach1979}
	z pewnymi uproszczeniami\cite{alg}
	i modyfikacją do obsługi
	ciągów numerowanych od \(0\)\cite{wazniakmimuw}
	oraz pokazano dowód jego poprawności
	i analizę złożoności obliczeniowej.
\end{abstract}
\tableofcontents
\pagebreak
\section{Definicja problemu}
Dane są dwa ciągi
\(A=(a_0,\dots,a_{n-1})\) oraz \(B=(b_0,\dots,b_{n-1})\)
długości \(n\).\linebreak
\(A\) i \(B\) są \emph{równoważne cyklicznie}
(\(A\equiv B\)),
gdy są równe w sensie list cyklicznych tzn.
\begin{cyclic_equivalence_1}[Równoważność cykliczna]
	\label{def:cyclic_equivalence_1}
	\[A\equiv B \iff \exists_{k_0\in\mathbb{Z}}
		\forall_{k\in\set{0,\dots,n-1}}\
		a_{(k_0+k)\Mod{n}}=b_k\]
\end{cyclic_equivalence_1}
\vskip 2em
Dla wygody dalszego zapisu oznaczmy:
\[a_k\coloneq a_{k\Mod{n}},\ b_k\coloneq b_{k\Mod{n}}\]
\vskip 1em
Zdefiniujmy \(A_k\) jako listę powstałą z przesunięcia cyklicznego
ciągu \(A\) takiego, że \(a_k\) jest
pierwszym elementem \(A_k\).
Analogicznie dla \(B_k\).\footnote{\(A_0=[a_0,\dots,a_{n-1}]\)
oraz \(B_0=[b_0,\dots,b_{n-1}]\)}
\begin{align*}
	A_k:=[a_{k},\dots,a_{n-1},a_0,\dots,a_{k-1}] \\
	B_k:=[b_{k},\dots,b_{n-1},b_0,\dots,b_{k-1}]
\end{align*}

\hyperref[def:cyclic_equivalence_1]
{Definicję \ref*{def:cyclic_equivalence_1}}
można przedstawić równoważnie jako:
\begin{cyclic_equivalence_2}[Równoważność cykliczna]
	\[A\equiv B \iff \exists_{k_0\in\mathbb{Z}}
		\ A_{k_0}=B\]
\end{cyclic_equivalence_2}
\vskip 2em
\begin{b_shift_def_remark}
	Oczywiście \(A\equiv B\) także wtedy, gdy
	\[\exists_{k_0,l_0\in\mathbb{Z}}
		\forall_{k\in\set{0,\dots,n-1}}\
		a_{k_0+k}=b_{l_0+k}\]
	lub
	\[\exists_{k_0,l_0\in\mathbb{Z}}
		\ A_{k_0}=B_{l_0}\]
\end{b_shift_def_remark}
\pagebreak
\section{Algorytm naiwny} \label{sec:naive_alg}
Można łatwo zauważyć, że
\begin{lemma_k0_values}
	\label{lem:lemma_k0_values}
	Jeżeli nie istnieje
	\(k_0\in\set{0,\dots, n-1}\) spełniające warunek:
	\[\forall_{k\in\set{0,\dots,n-1}}\ a_{k_0+k}=b_k\]
	to nie istnieje \(k_0\in\mathbb{Z}\) spełniające ten warunek.
\end{lemma_k0_values}
\begin{proof}
	Oczywiste.
\end{proof}
\begin{less_k0_values_to_check}
	\label{cor:less_k0_values_to_check}
	Żeby ustalić istnienie \(k_0\) z
	\hyperref[def:cyclic_equivalence_1]
	{Definicji \ref*{def:cyclic_equivalence_1}},
	wystarczy sprawdzić, czy
	\[\exists_{k_0\in\set{0,\dots,n-1}}
		\forall_{k\in\set{0,\dots,n-1}}\ a_{k_0+k}=b_k\]
\end{less_k0_values_to_check}
\vskip 2em
\subsection{Algorytm}
Algorytm naiwny sprawdza dla każdego \(l\in\set{0,\dots,n-1}\)
czy \[\forall_{k\in\set{0,\dots,n-1}}\ a_{l+k}=b_k\]
Jeżeli trafi na \(l\) spełniające warunek
to mamy \(k_0=l\)
i algorytm zwraca \(A\equiv B\),
w przeciwnym wypadku zwraca \(A\not\equiv B\).
Algorytm ma złożoność kwadratową\cite{wazniakmimuw}.
\subsection{Implementacja}
\inputminted{python3}{../implementacje/naive.py}
\pagebreak
\section{Algorytm wykorzystujący wyszukiwanie wzorca}
Utwórzmy listę \[AA:=[a_0,\dots, a_{n-1}, a_0,\dots, a_{n-1}]\]
i zauważmy, że dowolny spójny podciąg \(AA\) o długości \(n\)
rozpoczynający się od indeksu \(k\) ma postać \(A_k\),
czyli jest przesunięciem cyklicznym ciągu \(A\).
\begin{A_i_B_j_all_rotations}
	\label{cor:A_i_B_j_all_rotations}
	Łatwo zauważyć, że
	\[\set{A_0,\dots,A_{n-1}}=
		\set{A_\mathbb{Z}}\]
\end{A_i_B_j_all_rotations}
Zatem jeżeli sprawdzimy, czy \(B\) jest spójnym podciągiem \(AA\)
to otrzymamy rozwiązanie problemu równoważności cyklicznej.
\subsection{Algorytm}
Problem sprowadza się więc do problemu wyszukiwania wzorca,
który można rozwiązać wykorzystując np.
algorytm Knutha-Morrisa-Pratta.
Wykonuje on dla sprawdzania równoważności cyklicznej około
\(5n\) porównań\cite{shiloach1979}.
Wymaga on jednak dodatkowej pamięci (liniowa złożoność pamięciowa).
\subsection{Implementacja}
\inputminted[fontsize=\normalsize]{python3}
{../implementacje/pattern_matching.py}
\pagebreak
\section[Algorytm szybki]{Algorytm szybki\footnote{
	  Będę używał nieco innych oznaczeń niż w \cite{shiloach1979}
  }}
Algorytm wymaga istnienia porządku liniowego na zbiorze
\[\set{a_0,\dots,a_{n-1},b_0,\dots,b_{n-1}}\]
Niech \(a\le b\) oznacza, że \(a\) poprzedza
lub jest równe \(b\) w tym porządku, a \(a< b\) oznacza
\(x\le b \land a\neq b\).

Wtedy porządek leksykograficzny
na zbiorze \[\set{A_0,\dots,A_{n-1},B_0,\dots,B_{n-1}}\]
jest dobrym porządkiem.\\
Niech \(A_i\preceq B_j\) oznacza, że \(A_i\) poprzedza
lub jest równe \(B_j\) w porządku leksykograficznym,
a \(A_i\prec B_j\) oznacza \(A_i\preceq B_j \land A_i\neq B_j\).

Niech
\[A_{i_0}:=\min(\set{A_0,\dots,A_{n-1}})\
	\text{tzn.}\ \forall_{i\in\set{0,\dots,n-1}}\
	A_{i_0}\preceq A_i\]
analogicznie
\[B_{j_0}:=\min(\set{B_0,\dots,B_{n-1}})\]
\begin{equiv_iff_shift_sets_equal}
	Łatwo można zauważyć, że
	\[A\equiv B \iff
		\set{A_0,\dots,A_{n-1}}=\set{B_0,\dots,B_{n-1}}\]
\end{equiv_iff_shift_sets_equal}
\begin{lemma_equiv_if_lex_min_eq}
	\label{lem:equivalence_if_lex_min_eq}
	\[A\equiv B \iff A_{i_0}=B_{j_0}\]
\end{lemma_equiv_if_lex_min_eq}
\begin{proof}
	Oczywiste.
\end{proof}
\begin{lemma_k_shift_greater}
	\label{lem:k_shift_greater}
	\[\forall_{l\in\set{0,\dots,k-1}}\ a_{i+l}=b_{j+l}
		\land a_{i+k}<b_{j+k}
		\implies \forall_{l\in\set{0,\dots,k}}
		A_{i+l}\prec B_{j+l}\]
\end{lemma_k_shift_greater}
\begin{proof}
	Trywialne.
\end{proof}
Zdefiniujmy
\[D_A:=\set{i:\exists_j\ B_j\prec A_i}\]
analogicznie
\[D_B:=\set{j:\exists_i\ A_i\prec B_j}\]
tzn. \(D_A\) to zbiór indeksów
wyznaczających takie przesunięcia cykliczne \(A\),
dla których istnieje dowolne przesunięcie cykliczne \(B\)
poprzedzające je w porządku leksykograficznym.
Analogicznie dla \(B\).
\pagebreak

Zdefiniujmy
\[i_D:=\begin{cases}
		-1        & \text{jeżeli}\ D_A=\emptyset  \\
		\max(D_A) & \text{w przeciwnym przypadku}
	\end{cases}\]
\[j_D:=\begin{cases}
		-1        & \text{jeżeli}\ D_B=\emptyset  \\
		\max(D_B) & \text{w przeciwnym przypadku}
	\end{cases}\]
\begin{lemma_1_n_in_d_no_equiv}
	\label{lem:1_n_in_d_no_equiv}
	\[D_A\supseteq\set{0,\dots,n-1}\lor
		D_B\supseteq\set{0,\dots,n-1}
		\implies A\not\equiv B\]
\end{lemma_1_n_in_d_no_equiv}
\begin{proof}
	Dowód wynika z
	\hyperref[lem:equivalence_if_lex_min_eq]
	{Lematu \ref*{lem:equivalence_if_lex_min_eq}}
	\[A\equiv B\implies i_0\notin D_A \land j_0\notin D_B\]
\end{proof}
\subsection{Algorytm}
Zasadą działania algorytmu jest szybkie zbieranie indeksów
przesunięć cyklicznych, które powinny być
w zbiorach \(D_A\) i \(D_B\),\footnote{
	W implementacji te zbiory są niejawne.
}
dopóki nie zajdzie jeden z poniższych warunków stopu:
\begin{enumerate}[label=(\arabic*)]
	\item \label{term_cond_1} \(D_A\supseteq\set{0,\dots,n-1}
	      \lor D_B\supseteq\set{0,\dots,n-1}\)
	\item \label{term_cond_2}Znajdziemy takie \(i\) oraz \(j\),
	      że \(\forall_{k\in\set{0,\dots,n-1}}\ a_{i+k}=b_{j+k}\)
	      (tzn. \(A\equiv B\))
\end{enumerate}
Głównym elementem algorytmu będzie procedura \textsc{compare},
która znajduje pierwszy indeks \(k\),
dla którego \(a_{i+k}\neq b_{j+k}\)
i w zależności od tego, czy
\(a_{i+k}<b_{j+k}\)
dodaje do \(D_B\) zbiór \(\set{j,\dots,j+k}\)
albo do \(D_A\) zbiór \(\set{i,\dots,i+k}\)
(Patrz \cref{lem:k_shift_greater}).
Jeżeli nie znajdzie nierówności to zwraca \(A\equiv B\).

Procedura \textsc{rowncykl} wykonuje procedurę
\textsc{compare}, dopóki nie zajdzie jeden z powyższych warunków.
\pagebreak

\begin{algorithmic}
	\State \textbf{global} \(D_a, D_B, i_D, j_D\)
	\Procedure{compare}{$i, j$}
	\For{$k=0,\dots,n-1$}
	\If{\(a_{i+k}\neq b_{j+k}\)\footnote{W analizie
			złożoności będziemy liczyć te porównania}}
	\If{\(a_{i+k}<b_{j+k}\)}
	\State \(D_b\gets D_b\cup\set{j,\dots,j+k}\)
	\State \(j_D\gets j+k\)
	\Else
	\State \(D_a\gets D_a\cup\set{i,\dots,i+k}\)
	\State \(i_D\gets i+k\)
	\EndIf
	\State \Return
	\EndIf
	\EndFor
	\State \Return{\(A\equiv B\)}
	\EndProcedure
	\Procedure{rowncykl}{$A, B$}
	\State \(D_a\gets \emptyset\)
	\State \(D_b\gets \emptyset\)
	\State \(i_D\gets -1\)
	\State \(j_D\gets -1\)
	\While{\(D_A\nsupseteq\set{0,\dots,n-1}\land
		D_B\nsupseteq\set{0,\dots,n-1}\)}
	\If{\Call{compare}{$i_D+1,j_D+1$} returns \(A\equiv B\)}
	\State \Return{\(A\equiv B\)}
	\EndIf
	\EndWhile
	\State \Return{\(A\not\equiv B\)}
	\EndProcedure
\end{algorithmic}
\pagebreak
\subsection{Poprawność}
\begin{lemma_niezmiennik}[Niezmiennik]
	\label{lem:niezmiennik}
	Po każdym wykonaniu procedury \textnormal{\textsc{compare}}
	\(D_A\) oraz \(D_B\) mają postać
	\[D_A=\set{0,\dots,i_D}\]
	\[D_B=\set{0,\dots,j_D}\]
	tzn. w \(D_A\) nie brakuje żadnego elementu
	między \(0\) a \(i_D\), analogicznie w \(D_B\)
	nie brakuje
	żadnego elementu między \(0\) a \(j_D\).
\end{lemma_niezmiennik}
\begin{proof}
	Dowód wynika trywialnie z faktu, że
	jedyny sposób, w jaki modyfikowane jest \(D_A\)
	to \[D_a\gets D_a\cup\set{i_D+1,\dots,i_D+k}\]
	analogicznie dla \(D_B\)
	\[D_b\gets D_b\cup\set{j_D+1,\dots,j_D+k}\]
\end{proof}
\begin{not_storing_sets}
	\label{cor:not_storing_sets}
	Z \hyperref[lem:niezmiennik]{Lematu \ref*{lem:niezmiennik}}
	wynika, że nie musimy przechowywać zbiorów \(D_A\) i \(D_B\),
	wystarczy nam jedynie \(i_D\) oraz \(j_D\).
\end{not_storing_sets}
\begin{termination_theorem}[Własność stopu]
	\label{thm:termination}
	Procedura \textnormal{\textsc{rowncykl}}
	zatrzymuje się w skończonym czasie dla wszystkich
	poprawnych danych wejściowych.
\end{termination_theorem}
\begin{proof}
	Jeżeli dla jakiegoś \(i\) oraz \(j\) procedura
	\textsc{compare} zwróci \(A\equiv B\) to
	\textsc{rowncykl} się zatrzymuje.

	Jeżeli procedura \textsc{compare} nie zwróci \(A\equiv B\)
	dla żadnego \(i\) oraz \(j\) to z każdym wywołaniem
	\textsc{compare} \(i_D+j_D\) rośnie co najmniej o \(1\),
	więc w końcu
	(z \hyperref[lem:niezmiennik]{Lematu \ref*{lem:niezmiennik}})
	\hyperref[term_cond_1]{warunek stopu 1}
	zostanie spełniony i procedura
	\textsc{rowncykl} się zatrzyma.
\end{proof}
\begin{correctness_theorem}[Poprawność]
	\label{thm:correctness}
	Procedura \textnormal{\textsc{rowncykl}}
	zwraca \(A\equiv B\) wtedy i tylko wtedy
	gdy \(A\equiv B\).
\end{correctness_theorem}
\begin{proof}
	Jeżeli procedura \textsc{rowncykl} zwróciła \(A\equiv B\)
	to znaczy, że procedura \textsc{compare}
	wykazała równość dwóch przesunięć cyklicznych \(A\) i \(B\),
	a więc \(A\equiv B\).
	Jeżeli procedura \textsc{rowncykl} nie zwróci \(A\equiv B\)
	to zwróci \(A\not\equiv B\), gdyż z
	\hyperref[thm:termination]{Twierdzenia \ref*{thm:termination}}
	wiemy, że zawsze się zatrzymuje.
	Zwrócenie \(A\not\equiv B\) oznacza, że wystąpił
	\hyperref[term_cond_1]{warunek stopu 1},
	więc z \hyperref[lem:1_n_in_d_no_equiv]
	{Lematu \ref*{lem:1_n_in_d_no_equiv}} \(A\not\equiv B\).
\end{proof}
\pagebreak
\subsection{Złożoność}
\begin{def_success_compare}
	\label{def:success_compare}
	Wykonanie procedury \textnormal{\textsc{compare}}
	nazwiemy \emph{udanym}, jeżeli zwróci \(A\equiv B\).
	(Wykonanie, które nie jest udane jest \emph{nieudane})
\end{def_success_compare}
\begin{def_D_star}
	\label{def:D_star}
	Oznaczmy \(D_A\) oraz \(D_B\)
	bezpośrednio przed ostatnim wykonaniem
	procedury \textnormal{\textsc{compare}}
	jako odpowiednio \(D_A^*\) oraz \(D_B^*\).
\end{def_D_star}
\begin{lemma_unsuccesful_comparisons}
	\label{lem:unsuccesful_comparisons}
	Łączna liczba porównań wykonanych przez
	wywołania procedury \textnormal{\textsc{compare}},
	z wyjątkiem ostatniego wynosi
	\(\abs{D_A^*}+\abs{D_B^*}\)
\end{lemma_unsuccesful_comparisons}
\begin{proof}
	Każde wywołanie procedury \textsc{compare}
	z wyjątkiem ostatniego jest nieudane
	i zwiększa \(\abs{D_A}+\abs{D_B}\)
	o dokładnie tyle ile wykonało porównań.
	Wynika to trywialnie z definicji
	procedury \textsc{compare}.
\end{proof}
\begin{lemma_d_star_len}
	\label{lem:d_star_len}
	\(\abs{D_A^*}+\abs{D_B^*}\leq2n-2\)
\end{lemma_d_star_len}
\begin{proof}
	Jeżeli \(\abs{D_A^*}+\abs{D_B^*}\ge2n-1\)
	to oznacza, że
	\[\set{0,\dots,n-1}\subset D_A^* \lor
		\set{0,\dots,n-1}\subset D_B^*\]
	co nie jest możliwe (patrz \cref{def:D_star}).
\end{proof}
\begin{complexity_theorem}
	Algorytm wykonuje co najwyżej \(3n-2\) porównań.
\end{complexity_theorem}
\begin{proof}
	Dowód wynika z
	\hyperref[lem:unsuccesful_comparisons]
	{Lematu \ref*{lem:unsuccesful_comparisons}}
	oraz
	\hyperref[lem:d_star_len]{Lematu \ref*{lem:d_star_len}},
	gdyż ostatnie wywołanie procedury \textsc{compare}
	może wykonać co najwyżej \(n\) porównań.
\end{proof}
\begin{o_complexity}
	Algorytm ma pesymistyczną złożoność czasową \(\mathcal{O}(n)\)
	i dodatkową złożoność pamięciową \(\mathcal{O}(1)\)
	(z \hyperref[cor:not_storing_sets]
	{Wniosku \ref*{cor:not_storing_sets}})
\end{o_complexity}
\begin{remark_max_comp_input}
	Największa ilość porównań występuje dla
	ciągów postaci\\
	\(A=(1,1,\dots,1,2,0,1),
	B=(1,1,1,\dots,1,2,0)\) \cite{alg}
\end{remark_max_comp_input}
\pagebreak
\subsection{Implementacja}
\inputminted{python3}{../implementacje/fast.py}
\pagebreak
\subsection{Przykłady wykonania}
Listy wypisane z przesunięciem
\((i+1)\) dla \(A\) i \((j+1)\) dla \(B\)
\vskip 1em
	{
		\centering
		\begin{tabular}[t]{|c|}
			\hline
			\(n=8\)                                      \\
			\(A=(1,2,1,2,1,1,2,1)\)                      \\
			\(B=(2,1,1,2,1,1,2,1)\)                      \\
			\hline
			\(i=-1,j=-1\)                                \\
			\(A_0=[\textcolor{red}{1},2,1,2,1,1,2,1]\)   \\
			\(B_0=[\textcolor{Green}{2},1,1,2,1,1,2,1]\) \\
			\(k=1, A_0\prec B_0\)                        \\
			\hline
			\(i=-1,j=0\)                                 \\
			\(A_0=[1,\textcolor{Green}{2},1,2,1,1,2,1]\) \\
			\(B_1=[1,\textcolor{red}{1},2,1,1,2,1,2]\)   \\
			\(k=2, B_1\prec A_0\)                        \\
			\hline
			\(i=1,j=0\)                                  \\
			\(A_2=[1,\textcolor{Green}{2},1,1,2,1,1,2]\) \\
			\(B_1=[1,\textcolor{red}{1},2,1,1,2,1,2]\)   \\
			\(k=2, B_1\prec A_2\)                        \\
			\hline
			\(i=3,j=0\)                                  \\
			\(A_4=[1,1,2,1,1,2,1,2]\)                    \\
			\(B_1=[1,1,2,1,1,2,1,2]\)                    \\
			brak nierówności                             \\
			\(A\equiv B\)                                \\
			\hline
		\end{tabular}
		\quad
		\begin{tabular}[t]{|c|}
			\hline
			\(n=10\)                                         \\
			\(A=(1,1,2,1,3,1,1,2,1,1)\)                      \\
			\(B=(1,1,1,2,1,1,1,2,1,3)\)                      \\
			\hline
			\(i=-1,j=-1\)                                    \\
			\(A_0=[1,1,\textcolor{Green}{2},1,3,1,1,2,1,1]\) \\
			\(B_0=[1,1,\textcolor{red}{1},2,1,1,1,2,1,3]\)   \\
			\(k=3, B_0\prec A_0\)                            \\
			\hline
			\(i=2,j=-1\)                                     \\
			\(A_3=[1,\textcolor{Green}{3},1,1,2,1,1,1,1,2]\) \\
			\(B_0=[1,\textcolor{red}{1},1,2,1,1,1,2,1,3]\)   \\
			\(k=2, B_0\prec A_3\)                            \\
			\hline
			\(i=4,j=-1\)                                     \\
			\(A_5=[1,1,\textcolor{Green}{2},1,1,1,1,2,1,3]\) \\
			\(B_0=[1,1,\textcolor{red}{1},2,1,1,1,2,1,3]\)   \\
			\(k=3, B_0\prec A_5\)                            \\
			\hline
			\(i=7,j=-1\)                                     \\
			\(A_8=[1,1,1,\textcolor{red}{1},2,1,3,1,1,2]\)   \\
			\(B_0=[1,1,1,\textcolor{Green}{2},1,1,1,2,1,3]\) \\
			\(k=4, A_8\prec B_0\)                            \\
			\hline
			\(i=7,j=3\)                                      \\
			\(A_8=[1,1,1,\textcolor{red}{1},2,1,3,1,1,2]\)   \\
			\(B_4=[1,1,1,\textcolor{Green}{2},1,3,1,1,1,2]\) \\
			\(k=4, A_8\prec B_4\)                            \\
			\hline
			\(i=7,j=7\)                                      \\
			\(A_8=[1,\textcolor{red}{1},1,1,2,1,3,1,1,2]\)   \\
			\(B_8=[1,\textcolor{Green}{3},1,1,1,2,1,1,1,2]\) \\
			\(k=2, A_8\prec B_8\)                            \\
			\hline
			\(i=7,j=9\)                                      \\
			\(j\ge n-1\), więc \(A\not\equiv B\)             \\
			\hline
		\end{tabular}
		\par
	}
\pagebreak
\printbibliography[heading=bibintoc]
\end{document}
